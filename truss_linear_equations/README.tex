\documentclass{article}
\usepackage[utf8]{inputenc}

\title{CME 211 Homework 4 README}
\author{tlmenest }
\date{October 2021}

\begin{document}

\maketitle

\section{Introduction}

The goal of the Truss class is to analyze the geometry of a truss, which is 
given by two files, joints.dat and beams.dat. The first one gives the x and y coordinate positions of each joint, along with the forces in x and y direction and whether or not the point is rigidly supported with zero displacements. The second file gives the joints connecting each beam. The class will have as input the path of those files, assign them as attributes to the Truss object, make a plot of the geometry of the Truss and compute its static equilibrium.

\section{Constructor}

The constructor of the class takes two inputs: the paths of the joints and beams files and assigns them as attributes to the Truss object.

\section{Representation}

The repr method tells the code to print the final string attribute of a Truss object when directly printing out a Truss object.

\section{Compute static equilibrium}

The compute static equilibrium method will read both the beams and joints file, determine the normalized vector of forces for each joint and, from this, build the equations that will be added to the CSR matrix. Finally, it creates a B vector (i.e. the solutions) equal to the forces at each joint times minus one. Finally, the method will solve the linear system of equations and build an output string with the value of each beam force. This value will be the new value of the attribute final string of the Truss object, which will be printed when calling the repr method.

\section{Plot geometry}

This method will read the beams and joints files, build a tuple to store the connections between each point and iterate over it to draw a line between each points that are connected. Finally, it outputs the plot.

\end{document}

