\documentclass{article}
\usepackage{geometry}
\usepackage{graphicx}
\usepackage{algorithm}
\usepackage{algorithmic}
\geometry{left=3cm, right=3cm, top=2cm, bottom=2cm}
\title{Write-up: Project part 1}
\author{Thomas Le Menestrel}
\date{\today}
\setlength{\parindent}{0pt}\begin{document}
\maketitle

\begin{enumerate}
\item \textbf{Pseudo-code for Conjugate Gradient (CG) algorithm}
\begin{algorithm}
\caption{CG algorithm} 
\begin{algorithmic}
    \STATE \textbf{Initialize}: $u_0$
    \STATE $r_0=b-Au_0$
    \STATE $p_0=r_0$
    \STATE $niter=0$
    \WHILE{$niter<nitermax$}
        \STATE $niter=niter + 1$
        \STATE $\alpha_n={r_n}^Tr_n/{p_n}^TAp_n$
        \STATE $u_{n+1}=u_n+\alpha_np_n$
        \STATE $r_{n+1}=r_n-\alpha_nAp_n$
        \IF{$\left\|r_{n+1}\right\|_2/\left\|r_0\right\|_2<threshold$}
            \STATE \textbf{break}
        \ENDIF
        \STATE $\beta_n={r_{n+1}}^Tr_{n+1}/{r_n}^Tr_n$
        \STATE $p_{n+1}=r_{n+1}+\beta_np_n$
    \ENDWHILE
\end{algorithmic}
\end{algorithm}

\bigskip
\noindent
\smallskip
\noindent
\textbf{Answer}:
\smallskip

The CG algorithm allows to solve Ax = b when A is sparse and with initial guess x. To avoid redundancy and use functions, I first wrote down the major steps of the algorithms, which were the following:

\begin{itemize}
\item Add or substract vectors
\item Multiply a vector by a scalar
\item Dot product of two vectors
\item 2-norm of a vector
\item CSR matrix-vector multiplication
\end{itemize}

Based on this, I implemented those operations to build the CG algorithm using 5 different C++ functions, which are:
\begin{itemize}
\item addVecsWithCoef(vec1, vec2, coef), which returns x + coef * y
\item multiplyVecByScalar(vec, scalar) to multiply vec by a scalar
\item dotPrd(x), which computes the dot product of two vectors
\item L2Norm(vec), which computes the L2 norm of a vector
\item csr\_mat\_vec\_product(A, x), which computes the matrix-vector multiplication Ax 
\end{itemize}

\end{document}
